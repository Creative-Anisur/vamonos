\documentclass[12pt]{article}

\usepackage{mathptmx,amssymb,amsmath}
\usepackage{fullpage}
\usepackage{enumerate}
\usepackage{fancyvrb}


%%%
%%% Formatting details
%%%
\sloppy
\sloppypar
\widowpenalty=0
\clubpenalty=0
\displaywidowpenalty=0
\raggedbottom
\pagestyle{plain}


% \topskip0pt
% \parskip0pt
% \partopsep0pt

\DefineVerbatimEnvironment{program}{Verbatim}
  {baselinestretch=1.0,xleftmargin=5mm,fontsize=\small,samepage=true}

\def\denseitems{
    \itemsep1pt plus1pt minus1pt
    \parsep0pt plus0pt
    \parskip0pt\topsep0pt}


\newcommand{\prog}[1]{{\small\texttt{#1}}}
% \newcommand{\bs}{\texttt{\symbol{92}}}

\begin{document}

\title{\textbf{Vamonos}
\\ Algorithm Visualization Framework}

\author{Mike Rosulek and Brent Carmer \\
School of EECS \\
Oregon State University
}

\maketitle

\section{Introduction}
\label{sec:intro}

Vamonos is an interactive algorithm visualization framework for the browser. It allows easy creation
of algorithm visualizations. It supports a number of data structures including dynamic display of
arrays and graphs. It supports recursion and can display the call-stack including argument and
return values.

\section{Users and Use-Cases}
\label{sec:users}

We have two main users: instructors and students.

\begin{itemize}
\item Instructors\\

  Instructors can use Vamonos in class to demonstrate the execution of algorithms. The
  visualizations are minimalist, with little visual distraction, giving instructors freedom to use
  them as they like. Edge cases can be drawn into data structures easily.

\item Students\\

  Students can experiment with different input. They can customize breakpoints and watched variables
  as needed to further their understanding. We have created entire lessons demonstrating algorithmic
  concepts like dynamic programming for self-study.

\end{itemize}

\section{Cognitive Dimension Evaluation}
\label{sec:cogdim}

\section{Basic Objects}
\label{sec:objects}

\section{Interpretation and Analyses}
\label{sec:analysis}

\section{Implementation}
\label{sec:implementation}

We use \prog{jQuery} heavily for most visualization tasks and \prog{D3.js} for displaying graphs.

\begin{itemize}
\item Algorithm implementation\\
  The algorithm is implemented in Javascript. Talk about _, with, etc.

Widgets initialization

The structure of a Vamonos demo is three part. The layout of the visualization happens in
\prog{HTML}. Various divs are tagged with ids, which are used to identify them later from the
javascript part.
\end{itemize}

\section{Design Evolution}
\label{sec:evol}



\section{Related Projects}
\label{sec:related}

Python tutor

\section{Future Work}
\label{sec:future}

Encyclopedia of algorithms
Compile visualization from pseudocode

\end{document}
